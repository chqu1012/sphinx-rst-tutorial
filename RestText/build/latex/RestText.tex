% Generated by Sphinx.
\def\sphinxdocclass{report}
\documentclass[a4paper,10pt,ngerman]{sphinxmanual}
\usepackage[utf8]{inputenc}
\DeclareUnicodeCharacter{00A0}{\nobreakspace}
\usepackage[T1]{fontenc}
\usepackage[german]{babel}
\usepackage{times}
\usepackage[Sonny]{fncychap}
\usepackage{longtable}
\usepackage{sphinx}
\usepackage{multirow}


\title{RestText Documentation}
\date{12. 03. 2013}
\release{0.0.1}
\author{Dung Chau}
\newcommand{\sphinxlogo}{}
\renewcommand{\releasename}{Release}
\makeindex

\makeatletter
\def\PYG@reset{\let\PYG@it=\relax \let\PYG@bf=\relax%
    \let\PYG@ul=\relax \let\PYG@tc=\relax%
    \let\PYG@bc=\relax \let\PYG@ff=\relax}
\def\PYG@tok#1{\csname PYG@tok@#1\endcsname}
\def\PYG@toks#1+{\ifx\relax#1\empty\else%
    \PYG@tok{#1}\expandafter\PYG@toks\fi}
\def\PYG@do#1{\PYG@bc{\PYG@tc{\PYG@ul{%
    \PYG@it{\PYG@bf{\PYG@ff{#1}}}}}}}
\def\PYG#1#2{\PYG@reset\PYG@toks#1+\relax+\PYG@do{#2}}

\expandafter\def\csname PYG@tok@gd\endcsname{\def\PYG@tc##1{\textcolor[rgb]{0.63,0.00,0.00}{##1}}}
\expandafter\def\csname PYG@tok@gu\endcsname{\let\PYG@bf=\textbf\def\PYG@tc##1{\textcolor[rgb]{0.50,0.00,0.50}{##1}}}
\expandafter\def\csname PYG@tok@gt\endcsname{\def\PYG@tc##1{\textcolor[rgb]{0.00,0.27,0.87}{##1}}}
\expandafter\def\csname PYG@tok@gs\endcsname{\let\PYG@bf=\textbf}
\expandafter\def\csname PYG@tok@gr\endcsname{\def\PYG@tc##1{\textcolor[rgb]{1.00,0.00,0.00}{##1}}}
\expandafter\def\csname PYG@tok@cm\endcsname{\let\PYG@it=\textit\def\PYG@tc##1{\textcolor[rgb]{0.25,0.50,0.56}{##1}}}
\expandafter\def\csname PYG@tok@vg\endcsname{\def\PYG@tc##1{\textcolor[rgb]{0.73,0.38,0.84}{##1}}}
\expandafter\def\csname PYG@tok@m\endcsname{\def\PYG@tc##1{\textcolor[rgb]{0.13,0.50,0.31}{##1}}}
\expandafter\def\csname PYG@tok@mh\endcsname{\def\PYG@tc##1{\textcolor[rgb]{0.13,0.50,0.31}{##1}}}
\expandafter\def\csname PYG@tok@cs\endcsname{\def\PYG@tc##1{\textcolor[rgb]{0.25,0.50,0.56}{##1}}\def\PYG@bc##1{\setlength{\fboxsep}{0pt}\colorbox[rgb]{1.00,0.94,0.94}{\strut ##1}}}
\expandafter\def\csname PYG@tok@ge\endcsname{\let\PYG@it=\textit}
\expandafter\def\csname PYG@tok@vc\endcsname{\def\PYG@tc##1{\textcolor[rgb]{0.73,0.38,0.84}{##1}}}
\expandafter\def\csname PYG@tok@il\endcsname{\def\PYG@tc##1{\textcolor[rgb]{0.13,0.50,0.31}{##1}}}
\expandafter\def\csname PYG@tok@go\endcsname{\def\PYG@tc##1{\textcolor[rgb]{0.20,0.20,0.20}{##1}}}
\expandafter\def\csname PYG@tok@cp\endcsname{\def\PYG@tc##1{\textcolor[rgb]{0.00,0.44,0.13}{##1}}}
\expandafter\def\csname PYG@tok@gi\endcsname{\def\PYG@tc##1{\textcolor[rgb]{0.00,0.63,0.00}{##1}}}
\expandafter\def\csname PYG@tok@gh\endcsname{\let\PYG@bf=\textbf\def\PYG@tc##1{\textcolor[rgb]{0.00,0.00,0.50}{##1}}}
\expandafter\def\csname PYG@tok@ni\endcsname{\let\PYG@bf=\textbf\def\PYG@tc##1{\textcolor[rgb]{0.84,0.33,0.22}{##1}}}
\expandafter\def\csname PYG@tok@nl\endcsname{\let\PYG@bf=\textbf\def\PYG@tc##1{\textcolor[rgb]{0.00,0.13,0.44}{##1}}}
\expandafter\def\csname PYG@tok@nn\endcsname{\let\PYG@bf=\textbf\def\PYG@tc##1{\textcolor[rgb]{0.05,0.52,0.71}{##1}}}
\expandafter\def\csname PYG@tok@no\endcsname{\def\PYG@tc##1{\textcolor[rgb]{0.38,0.68,0.84}{##1}}}
\expandafter\def\csname PYG@tok@na\endcsname{\def\PYG@tc##1{\textcolor[rgb]{0.25,0.44,0.63}{##1}}}
\expandafter\def\csname PYG@tok@nb\endcsname{\def\PYG@tc##1{\textcolor[rgb]{0.00,0.44,0.13}{##1}}}
\expandafter\def\csname PYG@tok@nc\endcsname{\let\PYG@bf=\textbf\def\PYG@tc##1{\textcolor[rgb]{0.05,0.52,0.71}{##1}}}
\expandafter\def\csname PYG@tok@nd\endcsname{\let\PYG@bf=\textbf\def\PYG@tc##1{\textcolor[rgb]{0.33,0.33,0.33}{##1}}}
\expandafter\def\csname PYG@tok@ne\endcsname{\def\PYG@tc##1{\textcolor[rgb]{0.00,0.44,0.13}{##1}}}
\expandafter\def\csname PYG@tok@nf\endcsname{\def\PYG@tc##1{\textcolor[rgb]{0.02,0.16,0.49}{##1}}}
\expandafter\def\csname PYG@tok@si\endcsname{\let\PYG@it=\textit\def\PYG@tc##1{\textcolor[rgb]{0.44,0.63,0.82}{##1}}}
\expandafter\def\csname PYG@tok@s2\endcsname{\def\PYG@tc##1{\textcolor[rgb]{0.25,0.44,0.63}{##1}}}
\expandafter\def\csname PYG@tok@vi\endcsname{\def\PYG@tc##1{\textcolor[rgb]{0.73,0.38,0.84}{##1}}}
\expandafter\def\csname PYG@tok@nt\endcsname{\let\PYG@bf=\textbf\def\PYG@tc##1{\textcolor[rgb]{0.02,0.16,0.45}{##1}}}
\expandafter\def\csname PYG@tok@nv\endcsname{\def\PYG@tc##1{\textcolor[rgb]{0.73,0.38,0.84}{##1}}}
\expandafter\def\csname PYG@tok@s1\endcsname{\def\PYG@tc##1{\textcolor[rgb]{0.25,0.44,0.63}{##1}}}
\expandafter\def\csname PYG@tok@gp\endcsname{\let\PYG@bf=\textbf\def\PYG@tc##1{\textcolor[rgb]{0.78,0.36,0.04}{##1}}}
\expandafter\def\csname PYG@tok@sh\endcsname{\def\PYG@tc##1{\textcolor[rgb]{0.25,0.44,0.63}{##1}}}
\expandafter\def\csname PYG@tok@ow\endcsname{\let\PYG@bf=\textbf\def\PYG@tc##1{\textcolor[rgb]{0.00,0.44,0.13}{##1}}}
\expandafter\def\csname PYG@tok@sx\endcsname{\def\PYG@tc##1{\textcolor[rgb]{0.78,0.36,0.04}{##1}}}
\expandafter\def\csname PYG@tok@bp\endcsname{\def\PYG@tc##1{\textcolor[rgb]{0.00,0.44,0.13}{##1}}}
\expandafter\def\csname PYG@tok@c1\endcsname{\let\PYG@it=\textit\def\PYG@tc##1{\textcolor[rgb]{0.25,0.50,0.56}{##1}}}
\expandafter\def\csname PYG@tok@kc\endcsname{\let\PYG@bf=\textbf\def\PYG@tc##1{\textcolor[rgb]{0.00,0.44,0.13}{##1}}}
\expandafter\def\csname PYG@tok@c\endcsname{\let\PYG@it=\textit\def\PYG@tc##1{\textcolor[rgb]{0.25,0.50,0.56}{##1}}}
\expandafter\def\csname PYG@tok@mf\endcsname{\def\PYG@tc##1{\textcolor[rgb]{0.13,0.50,0.31}{##1}}}
\expandafter\def\csname PYG@tok@err\endcsname{\def\PYG@bc##1{\setlength{\fboxsep}{0pt}\fcolorbox[rgb]{1.00,0.00,0.00}{1,1,1}{\strut ##1}}}
\expandafter\def\csname PYG@tok@kd\endcsname{\let\PYG@bf=\textbf\def\PYG@tc##1{\textcolor[rgb]{0.00,0.44,0.13}{##1}}}
\expandafter\def\csname PYG@tok@ss\endcsname{\def\PYG@tc##1{\textcolor[rgb]{0.32,0.47,0.09}{##1}}}
\expandafter\def\csname PYG@tok@sr\endcsname{\def\PYG@tc##1{\textcolor[rgb]{0.14,0.33,0.53}{##1}}}
\expandafter\def\csname PYG@tok@mo\endcsname{\def\PYG@tc##1{\textcolor[rgb]{0.13,0.50,0.31}{##1}}}
\expandafter\def\csname PYG@tok@mi\endcsname{\def\PYG@tc##1{\textcolor[rgb]{0.13,0.50,0.31}{##1}}}
\expandafter\def\csname PYG@tok@kn\endcsname{\let\PYG@bf=\textbf\def\PYG@tc##1{\textcolor[rgb]{0.00,0.44,0.13}{##1}}}
\expandafter\def\csname PYG@tok@o\endcsname{\def\PYG@tc##1{\textcolor[rgb]{0.40,0.40,0.40}{##1}}}
\expandafter\def\csname PYG@tok@kr\endcsname{\let\PYG@bf=\textbf\def\PYG@tc##1{\textcolor[rgb]{0.00,0.44,0.13}{##1}}}
\expandafter\def\csname PYG@tok@s\endcsname{\def\PYG@tc##1{\textcolor[rgb]{0.25,0.44,0.63}{##1}}}
\expandafter\def\csname PYG@tok@kp\endcsname{\def\PYG@tc##1{\textcolor[rgb]{0.00,0.44,0.13}{##1}}}
\expandafter\def\csname PYG@tok@w\endcsname{\def\PYG@tc##1{\textcolor[rgb]{0.73,0.73,0.73}{##1}}}
\expandafter\def\csname PYG@tok@kt\endcsname{\def\PYG@tc##1{\textcolor[rgb]{0.56,0.13,0.00}{##1}}}
\expandafter\def\csname PYG@tok@sc\endcsname{\def\PYG@tc##1{\textcolor[rgb]{0.25,0.44,0.63}{##1}}}
\expandafter\def\csname PYG@tok@sb\endcsname{\def\PYG@tc##1{\textcolor[rgb]{0.25,0.44,0.63}{##1}}}
\expandafter\def\csname PYG@tok@k\endcsname{\let\PYG@bf=\textbf\def\PYG@tc##1{\textcolor[rgb]{0.00,0.44,0.13}{##1}}}
\expandafter\def\csname PYG@tok@se\endcsname{\let\PYG@bf=\textbf\def\PYG@tc##1{\textcolor[rgb]{0.25,0.44,0.63}{##1}}}
\expandafter\def\csname PYG@tok@sd\endcsname{\let\PYG@it=\textit\def\PYG@tc##1{\textcolor[rgb]{0.25,0.44,0.63}{##1}}}

\def\PYGZbs{\char`\\}
\def\PYGZus{\char`\_}
\def\PYGZob{\char`\{}
\def\PYGZcb{\char`\}}
\def\PYGZca{\char`\^}
\def\PYGZam{\char`\&}
\def\PYGZlt{\char`\<}
\def\PYGZgt{\char`\>}
\def\PYGZsh{\char`\#}
\def\PYGZpc{\char`\%}
\def\PYGZdl{\char`\$}
\def\PYGZhy{\char`\-}
\def\PYGZsq{\char`\'}
\def\PYGZdq{\char`\"}
\def\PYGZti{\char`\~}
% for compatibility with earlier versions
\def\PYGZat{@}
\def\PYGZlb{[}
\def\PYGZrb{]}
\makeatother

\begin{document}

\maketitle
\tableofcontents
\phantomsection\label{index::doc}



\chapter{Listen}
\label{pages/listen:welcome-to-resttext-s-documentation}\label{pages/listen::doc}\label{pages/listen:listen}\begin{itemize}
\item {} 
This is a bulleted list.

\item {} 
It has two items, the second
item uses two lines.

\end{itemize}
\begin{enumerate}
\item {} 
This is a numbered list.

\item {} 
It has two items too.

\item {} 
This is a numbered list.

\item {} 
It has two items too.

\end{enumerate}
\begin{itemize}
\item {} 
this is

\item {} 
a list
\begin{itemize}
\item {} 
with a nested list

\item {} 
and some subitems

\end{itemize}

\item {} 
and here the parent list continues

\end{itemize}


\chapter{Formatierungen}
\label{pages/formatierungen:formatierungen}\label{pages/formatierungen::doc}
\emph{Text}
\textbf{text}
\code{text}


\chapter{Umbruch}
\label{pages/umbruch::doc}\label{pages/umbruch:umbruch}
\begin{DUlineblock}{0em}
\item[] These lines are
\item[] broken exactly like in
\item[] the source file.
\end{DUlineblock}


\chapter{SourceCode}
\label{pages/sourcecode::doc}\label{pages/sourcecode:sourcecode}
This is a normal text paragraph. The next paragraph is a code sample:

\begin{Verbatim}[commandchars=\\\{\}]
It is not processed in any way, except
that the indentation is removed.

It can span multiple lines.
\end{Verbatim}

This is a normal text paragraph again.


\chapter{Tabellen}
\label{pages/tabellen::doc}\label{pages/tabellen:tabellen}
Grid Tabelle

\begin{tabulary}{\linewidth}{|L|L|L|L|}
\hline
\textbf{
Header row, column 1
(header rows optional)
} & \textbf{
Header 2
} & \textbf{
Header 3
} & \textbf{
Header 4
}\\\hline

body row 1, column 1
 & 
column 2
 & 
column 3
 & 
column 4
\\\hline

body row 2
 & 
...
 & 
...
 & \\\hline
\end{tabulary}


Simple Tabelle

\begin{tabulary}{\linewidth}{|L|L|L|}
\hline
\textbf{
A
} & \textbf{
B
} & \textbf{
A and B
}\\\hline

False
 & 
False
 & 
False
\\\hline

True
 & 
False
 & 
False
\\\hline

False
 & 
True
 & 
False
\\\hline

True
 & 
True
 & 
True
\\\hline
\end{tabulary}



\chapter{Sections}
\label{pages/sections:sections}\label{pages/sections::doc}

\section{This is a 1. heading}
\label{pages/sections:this-is-a-1-heading}

\section{This is a 1. heading}
\label{pages/sections:id1}

\subsection{This is a heading}
\label{pages/sections:this-is-a-heading}

\subsubsection{This is a heading}
\label{pages/sections:id2}

\paragraph{This is a heading}
\label{pages/sections:id3}

\subparagraph{This is a heading}
\label{pages/sections:id4}
This is a heading

This is a heading


\paragraph{Title 1}
\label{pages/sections:title-1}
heute geht es eventuell anders
heute geht es eventuell anders
heute geht es eventuell anders
heute geht es eventuell anders
heute geht es eventuell anders


\paragraph{Title 2}
\label{pages/sections:title-2}
heute geht es eventuell anders
heute geht es eventuell anders
heute geht es eventuell anders
heute geht es eventuell anders
heute geht es eventuell anders


\subsubsection{subtitle 2.1}
\label{pages/sections:subtitle-2-1}

\subsubsection{subtitle}
\label{pages/sections:subtitle}

\chapter{Direktiven}
\label{pages/direktiven:direktiven}\label{pages/direktiven::doc}
\begin{notice}{danger}{Gefahr:}
Beware killer rabbits!
\end{notice}

\begin{notice}{attention}{Achtung:}
Beware killer rabbits!
\end{notice}

\begin{notice}{caution}{Vorsicht:}
Beware killer rabbits!
\end{notice}

\begin{notice}{error}{Fehler:}
Beware killer rabbits!
\end{notice}

\begin{notice}{hint}{Hinweis:}
Beware killer rabbits!
\end{notice}

\begin{notice}{note}{Bemerkung:}
Beware killer rabbits!
\end{notice}

\begin{notice}{important}{Wichtig:}
Beware killer rabbits!
\end{notice}

\begin{notice}{tip}{Tipp:}
Beware killer rabbits!
\end{notice}

\begin{notice}{warning}{Warnung:}
Beware killer rabbits!
\end{notice}

\includegraphics{DCIcon.png}


\chapter{Fussnoten}
\label{pages/fussnoten:fussnoten}\label{pages/fussnoten::doc}

\section{Normale Fussnoten}
\label{pages/fussnoten:normale-fussnoten}
Lorem ipsum \footnote{
Text of the first footnote.
} dolor sit amet ... \footnote{
Text of the second footnote.
}


\section{Citations}
\label{pages/fussnoten:citations}
Lorem ipsum {\hyperref[pages/fussnoten:zitat01]{{[}Zitat01{]}}} dolor sit amet.


\section{Replacement}
\label{pages/fussnoten:replacement}

\chapter{Indices and tables}
\label{index:indices-and-tables}\begin{itemize}
\item {} 
\emph{genindex}

\item {} 
\emph{search}

\end{itemize}

\begin{thebibliography}{Zitat01}
\bibitem[Zitat01]{Zitat01}{\phantomsection\label{pages/fussnoten:zitat01} 
Book or article reference, URL or whatever.
}
\end{thebibliography}



\renewcommand{\indexname}{Stichwortverzeichnis}
\printindex
\end{document}
